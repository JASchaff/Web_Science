\documentclass[12pt, letterpaper]{article}
\usepackage{listings}
\usepackage{graphicx}
\usepackage{color}
\usepackage{caption}
\usepackage{subcaption}
\usepackage{hyperref}
\usepackage{fancyhdr}
\usepackage{mathrsfs}
\usepackage[margin=3cm]{geometry}
\usepackage[dvips]{epsfig}
\usepackage{placeins}
\usepackage{longtable}
\usepackage{wrapfig}
\usepackage{caption}
\usepackage{alltt}

\setlength{\parindent}{0.0in}
\setlength{\parskip}{0.05in}

\definecolor{dkgreen}{rgb}{0,0.6,0}
\definecolor{gray}{rgb}{0.5,0.5,0.5}
\definecolor{mauve}{rgb}{0.58,0,0.82}
\definecolor{deepblue}{rgb}{0,0,0.7}
\definecolor{deepred}{rgb}{0.6,0,0}
\definecolor{deepgreen}{rgb}{0,0.5,0}
\definecolor{red}{rgb}{0.9,0,0}

\newcommand\course{CS532}
\newcommand\semester{Spring 2017}
\newcommand\hwnum{9}
\newcommand\yourname{Justin Schaffner}
\newcommand\login{JASchaff}
\newenvironment{answer}[1]{\subsection*{Problem #1}}

\pagestyle{fancyplain}
\headheight 40pt
\lhead{\yourname\ (\login)\\\course\ --- \semester}
\chead{\textbf{\Large Assignment \hwnum}}
\rhead{\today}
\headsep 40pt

\DeclareCaptionFont{white}{\color{white}}
\DeclareCaptionFormat{listing}{\colorbox[cmyk]{0.43, 0.35, 0.35,0.01}{\parbox{\dimexpr\textwidth-2\fboxsep\relax}{#1#2#3}}}
\captionsetup[lstlisting]{format=listing,labelfont=white,textfont=white, singlelinecheck=false, margin=0pt, font={bf,footnotesize}}
\captionsetup[figure]{format=listing,labelfont=white,textfont=white, singlelinecheck=false, margin=0pt, font={bf,footnotesize}}
\captionsetup[alltt]{format=listing,labelfont=white,textfont=white, singlelinecheck=false, margin=0pt, font={bf,footnotesize}}



\lstnewenvironment{MyBash}[1][]{\noindent\lstset{#1, language=bash, aboveskip=3mm, belowskip=3mm, showstringspaces=false, columns=flexible, basicstyle={\small\ttfamily}, numbers=none, numberstyle=\tiny\color{grey}, keywordstyle=\color{black}, commentstyle=\color{dkgreen}, stringstyle=\color{black}, breaklines=true, breakatwhitespace=true, tabsize=3, frame=tb }}{}
\lstnewenvironment{MyPython}[1][]{\noindent\lstset{#1, language=Python, aboveskip=3mm, belowskip=3mm, basicstyle=\small, otherkeywords={self}, keywordstyle=\color{deepblue}, emph={MyClass,__init__}, emphstyle=\color{deepred}, stringstyle=\color{deepgreen}, commentstyle=\color{red}, frame=tb, showstringspaces=false, breaklines=true }}{}
\lstnewenvironment{MyR}[1][]{\noindent\lstset{#1, language=R, aboveskip=3mm, belowskip=3mm, basicstyle=\small, breaklines=true, frame=tb}}{}

\begin{document}

\begin{answer}{1: Build the Data Set}
Question 1: Choose a blog or a newsfeed (or something similar with an Atom or RSS feed). Every student should do a unique feed, so please ``claim" the feed on the class email list (first come, first served). It should be on a topic or topics of which you are qualified to provide classification training data. Find something with at least 100 entries (or items if RSS). Create between four and eight different categories for the entries
in the feed. Download and process the pages of the feed as per the week 12 class slides. Be sure to upload the raw data (Atom or RSS) to your github account. Create a table with 100 rows. This is your ``ground truth" (or ``gold standard") data.\\

I had a really hard time finding a blog with 100 entries that also allowed pagination of the RSS feed. Most of the pages I found don't have rel=next links for their RSS feeds. Instead I gathered several blogs that were all about cycling and built my data set from that. The list of blogs I used can be found in cyclefile.txt. I used the script getRSS.py in Listing \ref{lst:getrss} to pull the RSS pages and store them in individual files. In order to make a list of the 100 entries I intend to use, I added the function ``downtohundred()", shown in Listing \ref{lst:downtohundred} to fishermethod.py. Fishermethod will hold all of the functions used in this assignment other than getRSS, while the main script calling the functions can be found in feedclassify.py. I created 7 different categories for the entries:\\ community,	recommendation-travel,	recommendation-gear,	recommendation-media,	opinion,	how-to,	news\\ The categories are pretty self-explanatory, except perhaps for community, which I use for personal stories and anecdotes. The first section of feedclassify.py deals with the ground truth. Once the groundtruth.txt file has been established, it won't ask for it on further iterations. feedclassify.py can be seen in Listing \ref{lst:feedclassify} at the end of this report along with fishermethod.py in Listing \ref{lst:fishermethod}. 




\begin{MyPython}[caption=Q1 getRSS.py, label=lst:getrss]
import requests

header={'User-Agent': 'Mozilla/5.0 (Macintosh; Intel Mac OS X 10_12_3) AppleWebKit/537.36 (KHTML, like Gecko) Chrome/56.0.2924.87 Safari/537.36', 'Accept-Language':'en'}
URL=[]
file=open('cyclefile.txt', 'r')
for line in file:
    URL.append(line.rstrip())
count=0
for url in URL:
    page=requests.get(url, headers=header, allow_redirects=True, stream=True)
    out=open('feeds/RSSfeed'+str(count)+'.xml', 'w')
    out.write(page.text)
    count+=1
\end{MyPython}

\begin{MyPython}[caption=Q1 downtohundred, label=lst:downtohundred]
def downtohundred(path):
    articles={}
    for file in os.listdir(path):
        filepath=os.path.join(path, file)
        d=feedparser.parse(filepath)
        for e in d.entries:
            if 'summary' in e:
                articles.setdefault(e.title, e.summary)
            else:
                articles.setdefault(e.title, e.description)
    return [(v, k) for v,k in articles.items()][:100]

\end{MyPython}

\end{answer}

\begin{answer}{2: Fisher Classifier}
Train the Fisher classifier on the first 50 entries (the ``training set"), then use the classifier to guess the classification of the next 50 entries (the ``test set"). Create a table with 50 rows. Assess the performance of your classifier in each of your categories by computing precision, recall, and F-measure. Use the ``macro-averaged" label based method, as per: http://stats.stackexchange.com/questions/21551/how-to-compute-precisionrecall- for-multiclass-multilabel-classification . For example, if you have 5 categories (e.g., 80s, metal, alternative, electronic, cover), you will compute precision, recall, and F-measure for each category, and then compute the average across the 5 categories.\\
For question 2 I split the list of 100 articles and sent them to my own version of the read() function called ``myread()" in fishermethod.py in Listing \ref{lst:fishermethod}. The main difference is the ability to handle files from multiple feeds. The entries in the files are compared to the list of 100 from groundtruth.txt so that only htose 100 are looked at. The trainer also pulls the categorie directly from the groundtruth file rather than by user entry. Once the Classifier has been trained on the first 50, it calculates its best guess for the remaining 50. The table can be seen in results5050.txt or an excerpt in Table \ref{5050}. For the Precision, Recall and F-measure I used the Pandas-ML Library to make a confusion matrix. The implementation is in the F\textunderscore P\textunderscore R() function in Listing \ref{lst:fishermethod}. The Pandas-ML library does most of the work for you and generates the TP, FP and FN when print\textunderscore stats() is called. The results can be seen in Table \ref{50FPR} along with the Macro-Avg. Even though there was a category for news, there are apparently no news articles in the second half of the data set, so news is not represented. The results aren't terribly good. This could be because some of the entries may fall into more than one category, or I could just be terrible at making up categories.

\begin{table}[!htbp]
\small
\centering
\caption{50/50 Results}
\label{5050}
\resizebox{\textwidth}{!}{
\begin{tabular}{lll}
Title	& Actual	& Predicted\\
California Basketpackin?	& opinion	& recommendations-gear\\
In Search of the Best Bike Bell: The Bikeway Shootout.	& recommendations-gear	& recommendations-gear\\
Trans Ecuador Mountain Bike Route: Dirt Road Version	& recommendations-travel		& recommendations-travel\\
Lost Coast North	& recommendations-media	& recommendations-media\\
New Santa Cruz Chameleon: Bikepacking-ready trail stud?	& recommendations-gear	& recommendations-gear\\
Salsa Warbird Review: An Underbiking truce.	& recommendations-gear	& recommendations-travel\\
Revelate Mag-Tank: Fit for Vikings	& recommendations-gear	& recommendations-gear\\
Baja Divide: A Film by Tales on Tyres	& recommendations-media	& recommendations-media\\
Trans-Cuba: La Ruta Mala	& community	& recommendations-travel
\end{tabular}
}
\end{table}

\begin{table}[!htbp]
\small
\centering
\caption{50/50 FPR}
\label{50FPR}
\resizebox{\textwidth}{!}{
\begin{tabular}{llll}
Categorie	&  Precision	&  Recall	&  F-measure\\
Community	&  0.20	&  0.14	&  0.17\\
How-to	&  0.00	&  0.00	&  0.00\\
Opinion	&  0.00	&  0.00	&  0.00\\
Gear	&  0.55	&  0.90	&  0.68\\
Travel	&  0.57	&  0.67	&  0.62\\
Media	&  0.80	&  0.80	&  0.80\\
Macro-Avg	&  0.35	&  0.42	&  0.38
\end{tabular}
}
\end{table}

\end{answer}

\begin{answer}{3: Fisher Classifier on 90 trainer items}
Repeat question 2, but use the first 90 entries to train your classifier and the last 10 entries for testing. I redivided the original list of 100 into lists of 90 and 10, then ran the same functions over the lists, but using a new database. The results are in results9010.txt or in Table \ref{9010} and the Precision, Recall and F-measure are in Table \ref{90FRP}. The Precision is higher, but the Recall is much lower, so the F-measure came out to be about the same. 
\begin{table}[!htbp]
\small
\centering
\caption{90/10 Results}
\label{9010}
\resizebox{\textwidth}{!}{
\begin{tabular}{lll}
Title	& Actual	& Predicted\\
3 Car Free Gravel Climbs in Southern California	&  recommendations-travel	&  how-to\\
3 Must-Do Rides in Mississippi	&  recommendations-travel	&  how-to\\
Moulton Review (How do the Moultons and Bromptons compare?!)	&  recommendations-gear	&  recommendations-gear\\
700c to 650b Conversion Project	&  how-to	&  recommendations-travel\\
Shop Visit: Gladys Bikes	&  community	&  community\\
Bicycling Across Mississippi Video Playlist	&  community	&  community\\
PLPTucson: Shipping Bikes with the AirCaddy Bike Box	&  recommendations-gear	&  recommendations-gear\\
The Gino Behind the Gino Mount	&  recommendations-gear	&  how-to\\
Norther Cycles ? Randonneuring Paradise	&  recommendations-gear	&  how-to\\
Visiting Rivelo PDX	&  community	&  how-to
\end{tabular}}
\end{table}

\begin{table}[!htbp]
\small
\centering
\caption{90/10 FPR}
\label{90FRP}
\resizebox{\textwidth}{!}{
\begin{tabular}{llll}
Categorie	&  Precision	&  Recall	&  F-measure\\
Community	&  1.00	&  0.67	&  0.80\\
How-to	&  0.00	&  0.00	&  0.00\\
Gear	&  1.00	&  0.50	&  0.67\\
Travel	&	0.00	&  0.00	&  0.00\\
Macro-Avg	&  0.50	&  0.29	&  0.36
\end{tabular}}
\end{table}

\end{answer}

\begin{answer}{4: 10-fold Cross Validation}
Rerun question 3, but with ``10-fold cross validation". What was the change, if any, in precision and recall (and thus F-Measure)?\\
The hardest part of Question four was the logic for making 10 different test sets out of the data. That can be seen in Listing \ref{lst:feedclassify} in the last section. The Results of the 10 runs can be seen in Table \ref{1010FRP}, and the raw output is in the text file TenofTen\textunderscore Output.txt.
The final Precision for the Fisher Classifier on this data set was 0.367. The Recall is 0.44 and the F-measure is 0.37. The Precision and Recall are closer to question 1 than question 2, but the F-measure seems to be fairly constant for all three Questions. It could be low due to my choice in categories, but it could also be just that some of the categories are represented less in this data set. Perhaps with a larger data set, with more equal representation between the categories, the classifier would be more accurate. 
\FloatBarrier
\begin{longtable}{llll}
\small
\caption{10-fold FPR}\\
\label{1010FRP}\\
0th Run&&&\\
Categorie	&  Precision	&  Recall	&  F-measure\\
Community	&  1.0	&  0.67	&  0.80\\
How-to	&  0.0	&  0.0	&  0.0\\
Gear	&  1.0	&  0.5	&  0.66\\
Travel	&  0.0	&  0.0	& 0.0\\
Macro-Avg	& 0.5	&  0.29	&  0.37\\
*	&	&	&\\
1st Run	&	&	&\\
Categorie	&  Precision	&  Recall	&  F-measure\\
How-to	&  0.0	&  0.0	&  0.0\\
Gear	&  0.10	&  1.0	&  0.18\\
Macro-Avg	&  0.05	&  0.5	&  0.09\\
*	&	&	&\\
2nd Run	&	&	&\\
Categorie	&  Precision	&  Recall	&  F-measure\\
Community	&  1	&  0.5	&  0.67\\
How-to	&  0.0	&  0.0	&  0.0\\
Gear	&  0.8	&  1.0	&  0.89\\
Media	&  1.0	&  1.0	&  1.0\\
Travel	&  0.5	&  1.0	&  1.0\\
Macro-Avg	&  0.66	&  0.7	&  0.64\\
*	&	&	&\\
3rd Run	&	&	&\\
Categorie	&  Precision	&  Recall	&  F-measure\\
Community	&  0.0	&  0.0	&  0.0\\
Gear	&  0.86	&  1.0	&  0.92\\
Media	&  1.0	&  1.0	&  1.0\\
Travel	&  1.0	&  1.0	&  1.0\\
Macro-Avg	&  0.71	&  0.75	&  0.73\\
*	&	&	&\\
4th Run	&	&	&\\
Categorie	&  Precision	&  Recall	&  F-measure\\
Community	&  0.0	&  0.0	&  0.0\\
Gear	&  0.71	&  1.0	&  0.83\\
Media	&  1.0	&  1.0	&  1.0\\
Travel	&  1.0	&  1.0	&  1.0\\
Macro-Avg	&  0.68	&  0.75	&  0.71\\
*	&	&	&\\
5th Run	&	&	&\\
Categorie	&  Precision	&  Recall	&  F-measure\\
Community	&  0.0	&  0.0	&  0.0\\
How-to	&  0.0	&  0.0	&  0.0\\
News	&  0.0	&  0.0	&  0.0\\
Gear	&  0.75	&  1.0	&  0.86\\
Media	&  0.25	&  0.5	&  0.33\\
Travel	&  1.0	&  1.0	&  1.0\\
Macro-Avg	&  0.33	&  0.42	&  0.36\\
*	&	&	&\\
6th Run	&	&	&\\
Categorie	&  Precision	&  Recall	&  F-measure\\
Community	&  0.0	&  0.0	&  0.0\\
News	&  0.0	&  0.0	&  0.0\\
Gear	&  0.67	&  1.0	&  0.8\\
Media	&  0.5	&  1.0	&  0.67\\
Travel	&  1.0	&  1.0	&  1.0\\
Macro-Avg	&  0.43	&  0.6	&  0.49\\
*	&	&	&\\
7th Run	&	&	&\\
Categorie	&  Precision	&  Recall	&  F-measure\\
Communtiy	&  0.0	&  0.0	&  0.0\\
News	&  0.0	&  0.0	&  0.0\\
Opinion	&  0.0	&  0.0	&  0.0\\
Gear	&  0.25	&  1.0	&  0.4\\
Macro-Avg	&  0.062	&  0.25	&  0.1\\
*	&  	&  	&\\
8th Run	& 	& 	&\\
Categorie	&  Precision	&  Recall	&  F-measure\\
Community	&  0.0	&  0.0	&  0.0\\
How-to	&  0.0	&  0.0	&  0.0\\
Opinion	&  0.0	&  0.0	&  0.0\\
Gear	&  0.25	&  0.25	&  0.25\\
Travel	&  0.0	&  0.0	&  0.0\\
Macro-Avg	&  0.05	&  0.05	&  0.05\\
*	&	&	&\\
9th Run	&	&	&\\
Community	&  0.0	&  0.0	&  0.0\\
How-to	&  1.0	&  0.4	&  0.57\\
Opinion	&  0.0	&  0.0	&  0.0\\
Gear	&  0.0	&  0.0	&  0.0\\
Travel	&  0.0	&  0.0	&  0.0\\
*	&	&	&\\
10-Fold Cross	&  	&  	&\\
	&  Precision	&  Recall	&  F-measure\\
	&  0.37	&  0.44	&  0.37
\end{longtable}
\FloatBarrier





\end{answer}


\begin{MyPython}[caption= Fisher Method, label=lst:fishermethod]
import re
import math
import feedparser
import requests
from bs4 import BeautifulSoup as bs
from sqlite3 import dbapi2 as sqlite
import os
import pandas_ml as pd

header={'User-Agent': 'Mozilla/5.0 (Macintosh; Intel Mac OS X 10_12_3) AppleWebKit/537.36 (KHTML, like Gecko) Chrome/56.0.2924.87 Safari/537.36', 'Accept-Language':'en'}



#creates a confusion matrix and outputs the resulting statistics to the terminal
def F_P_R(articles, classifier):
    categories=classifier.categories()
    values=list((k,v) for k,v in articles.values())
    pred=list(v for k,v in values)
    act=list(k for k,v in values)
    pred=[item.rstrip() for item in pred]
    act=[item.rstrip() for item in act]
    conf=pd.ConfusionMatrix(act,pred)
    conf.print_stats()
   
            
    
                      
#selects 100 items out of a list of feeds
def downtohundred(path):
    articles={}
    for file in os.listdir(path):
        filepath=os.path.join(path, file)
        d=feedparser.parse(filepath)
        for e in d.entries:
            if 'summary' in e:
                articles.setdefault(e.title, e.summary)
            else:
                articles.setdefault(e.title, e.description)
    return [(v, k) for v,k in articles.items()][:100]

    


# Takes a filename of URL of a blog feed and classifies the entries
def read(feed,classifier):
    # Get feed entries and loop over them
    f=feedparser.parse(feed)
    for entry in f.entries:
        if 'summary' in entry:
            text=entry.summary
        else:
            text=entry.description
        print('\n-----')
        # Print the contents of the entry
        print('Title: '+entry.title.encode('utf-8'))
        print('Published: '+entry.published.encode('utf-8'))
        print('\n'+text.encode('utf-8'))
        # Combine all the text to create one item for the classifier
        fulltext='%s\n%s\n%s' % (entry.title,entry.published,text)
        # Print the best guess at the current category
        print('Guess: '+str(classifier.classify(fulltext)))
        # Ask the user to specify the correct category and train on that
        cl=raw_input('Enter category: ')
        classifier.train(fulltext,cl)


#my take on the read function. takes a list of titles to be found in a file of feeds       
def myread(articles, path, classifier):
    entries=[]
    for file in os.listdir(path):
        feed=os.path.join(path, file)
        f=feedparser.parse(feed)
        for entry in f.entries:
            entries.append(entry)
    
    for entry in entries:
        if entry.title in articles.keys():
            if 'summary' in entry:
                text=entry.summary
            else:
                text=entry.description
            
            # Combine all the text to create one item for the classifier
            fulltext='%s\n%s' % (entry.title,text)
            #changed this to pull the training data from the articles
            cl=articles[entry.title]
            classifier.train(fulltext,cl)
        
def myclassify(articles, path, classifier):
    temparticles=dict(list(x for x in articles.items()))
    entries=[]
    #pulls the individual entries out of the sepearte feed files
    for file in os.listdir(path):
        feed=os.path.join(path, file)
        f=feedparser.parse(feed)
        for entry in f.entries:
            entries.append(entry)
    #creates a table of actual and predicted classifications, storing them in a file in results folder
    for entry in entries:
        if entry.title in temparticles.keys():
            if 'summary' in entry:
                text=entry.summary
            else:
                text=entry.description
            fulltext='%s\n%s' % (entry.title,text)
            best=classifier.classify(fulltext)
            temparticles[entry.title]=(temparticles[entry.title], best)
            
    if not os.path.exists('./results'):
        os.mkdirs('./results')
    file=os.path.join('./results/','results'.rstrip()+str(len(os.listdir('./results')))+'.txt')
    with open(file, 'w') as out:
        for key in temparticles.keys():
            print(key, *temparticles[key], sep='\t', end='\n', file=out)
     #creates a confusion matrix
    F_P_R(temparticles, classifier)
        


def getwords(doc):
    splitter=re.compile('\\W*')
    # Split the words by non-alpha characters
    words=[s.lower( ) for s in splitter.split(doc)
if len(s)>2 and len(s)<20]
    # Return the unique set of words only
    return dict([(w,1) for w in words])


class classifier:
    def __init__(self,getfeatures,filename=None):
        # Counts of feature/category combinations
        self.fc={}
        # Counts of documents in each category
        self.cc={}
        self.getfeatures=getfeatures

    # Increase the count of a feature/category pair
    def incf(self,f,cat):
        count=self.fcount(f,cat)
        if count==0:
            self.con.execute("insert into fc values ('%s','%s',1)"% (f,cat))
        else:
            self.con.execute("update fc set count=%d where feature='%s' and category='%s'"% (count+1,f,cat))

    # Increase the count of a category
    def incc(self,cat):
        count=self.catcount(cat)
        if count==0:
            self.con.execute("insert into cc values ('%s',1)" % (cat))
        else:
            self.con.execute("update cc set count=%d where category='%s'" % (count+1,cat))

    # The number of times a feature has appeared in a category
    def fcount(self,f,cat):
        res=self.con.execute('select count from fc where feature="%s" and category="%s"'%(f,cat)).fetchone( )
        if res==None:
            return 0
        else: return float(res[0])

    # The number of items in a category
    def catcount(self,cat):
        res=self.con.execute('select count from cc where category="%s"'%(cat)).fetchone( )
        if res==None:
            return 0
        else:
            return float(res[0])

    # The total number of items
    def totalcount(self):
        res=self.con.execute('select sum(count) from cc').fetchone( )
        if res==None:
            return 0
        return res[0]

    # The list of all categories
    def categories(self):
        cur=self.con.execute('select category from cc')
        return [d[0] for d in cur]

    def train(self,item,cat):
        features=self.getfeatures(item)
        # Increment the count for every feature with this category
        for f in features:
            self.incf(f,cat)
        # Increment the count for this category
        self.incc(cat)
        self.con.commit( )

    def fprob(self,f,cat):
        if self.catcount(cat)==0:
            return 0
        # The total number of times this feature appeared in this
        # category divided by the total number of items in this category
        return self.fcount(f,cat)/self.catcount(cat)

    def weightedprob(self,f,cat,prf,weight=1.0,ap=0.5):
        # Calculate current probability
        basicprob=prf(f,cat)
        # Count the number of times this feature has appeared in
        # all categories
        totals=sum([self.fcount(f,c) for c in self.categories( )])
        # Calculate the weighted average
        bp=((weight*ap)+(totals*basicprob))/(weight+totals)
        return bp

    def setdb(self,dbfile):
        self.con=sqlite.connect(dbfile)
        self.con.execute('create table if not exists fc(feature,category,count)')
        self.con.execute('create table if not exists cc(category,count)')




    

class fisherclassifier(classifier):
    def __init__(self,getfeatures):
        classifier.__init__(self,getfeatures)
        self.minimums={}
    
    def cprob(self,f,cat):
        # The frequency of this feature in this category
        clf=self.fprob(f,cat)
        if clf==0:
            return 0
        # The frequency of this feature in all the categories
        freqsum=sum([self.fprob(f,c) for c in self.categories( )])
        # The probability is the frequency in this category divided by
        # the overall frequency
        p=clf/(freqsum)
        return p

    def fisherprob(self,item,cat):
        # Multiply all the probabilities together
        p=1
        features=self.getfeatures(item)
        for f in features:
            p*=(self.weightedprob(f,cat,self.cprob))
            # Take the natural log and multiply by -2
            fscore=-2*math.log(p)
        # Use the inverse chi2 function to get a probability
        return self.invchi2(fscore,len(features)*2)

    def invchi2(self,chi,df):
        m = chi / 2.0
        sum = term = math.exp(-m)
        for i in range(1, df//2):
            term *= m / i
            sum += term
        return min(sum, 1.0)

    def setminimum(self,cat,min):
        self.minimums[cat]=min

    def getminimum(self,cat):
        if cat not in self.minimums:
            return 0
        return self.minimums[cat]

    def classify(self,item,default=None):
        # Loop through looking for the best result
        best=default
        max=0.0
        for c in self.categories( ):
            p=self.fisherprob(item,c)
            # Make sure it exceeds its minimum
            if p>self.getminimum(c) and p>max:
                best=c
                max=p
        return best
\end{MyPython}

\begin{MyPython}[caption= Feed Classify, label=lst:feedclassify]
import requests
from fishermethod import *
import feedparser
import os

articles={}
#establish ground truth, if not already done
if not os.path.isfile('./groundtruth.txt'):
    listarticles=downtohundred('./feeds')
    with open('groundtruth.txt', 'w') as gtout:
        print('Establishing Ground Truth: %d'% len(articles))
        for a in listarticles:
            print(a[0], a[1], sep=': ')
            cat=input('Enter category: ')
            print(a[0], cat, sep='\t', file=gtout)
            articles.setdefault(a[0],cat)

else:
    for line in open('groundtruth.txt', 'r'):
        (title, actual)=line.split('\t')
        articles.setdefault(title,actual)
        
#process for question 2
halfone=dict(list(articles.items())[:50])
halftwo=dict(list(x for x in articles.items() if x not in halfone.items()))
cl=fisherclassifier(getwords)
if not os.path.isfile('./cyclingdb.db'):
    cl.setdb('cyclingdb.db')
    myread(halfone, './feeds', cl)
else: cl.setdb('cyclingdb.db')
myclassify(halftwo, './feeds', cl)

#process for question 3
cl2=fisherclassifier(getwords)
halfninety=dict(list(articles.items())[:90])
halften=dict(list(x for x in articles.items() if x not in halfninety.items()))
if not os.path.isfile('./cyclingdb2.db'):
    cl2.setdb('cyclingdb2.db')
    myread(halfninety, './feeds', cl2)
else:cl2.setdb('cyclingdb2.db')
myclassify(halften, './feeds', cl2)

#process for question 4
tenoften={}
topcount=dict(list(x for x in articles.items()))
downcount=dict(list(x for x in articles.items())[:90])
downcounter=90
#breaks out ten distinct sub dictionaries from the main Articles dictionary.
for i in range(10):
    tenoften[i]=dict(list(x for x in topcount.items() if x not in downcount.items()))
    downcounter-=10
    downcount=dict(list(downcount.items())[:downcounter])
    topcount=dict(list(x for x in topcount.items() if x not in tenoften[i].items()))
print('TEN OF TEN: %d' % len(tenoften))

#builds a new training data set to send to myread() for each subset of Articles, then sends the Articles subset to classify
for i in range(10):
    cl3=fisherclassifier(getwords)
    datafile=str('tenoften'.rstrip() +str(i)+'.db')
    if not os.path.isfile(os.path.join('./',datafile)):
        cl3.setdb(datafile)
        temp={}
        for keys in tenoften.keys():
            for key,item in tenoften[keys].items():
                if key not in tenoften[i].keys():
                    temp[key]=item                  
        myread(temp, './feeds', cl3)
    else:cl3.setdb(datafile)
    print('Data Set %s' % str('tenoften'+ str(i)))
    myclassify(tenoften[i],'./feeds',cl3)
\end{MyPython}



\end{document}